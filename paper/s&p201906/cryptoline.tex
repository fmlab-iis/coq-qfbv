
\begin{figure*}
  \centering
  \begin{math}
    \begin{array}{rclclcl}
      \multicolumn{7}{c}{
      \begin{array}{rclcrcl}
        \clNum & ::= & \cdots \ |\ \mytextsf{-2} \ |\ \mytextsf{-1}
                     \ |\ \mytextsf{0} \ |\ \mytextsf{1} \ |\
                     \mytextsf{2} \ |\  \cdots
        & \hspace{1em} &
        \mathit{Type} & ::= & \mytextsf{bit}\ |\
                              \mytextsf{uint32}\ |\ \mytextsf{sint32}\ |\
                              \mytextsf{uint64}\ |\ \mytextsf{sint64}\ |\ \cdots
      \end{array}
      }\\
      \multicolumn{7}{c}{
      \begin{array}{rclcrclcrcl}
      \mathit{Const} & ::= & \clNum\mytextsf{@}\mathit{Type}
      & \hspace{1em} &
      \clVar & ::= & \cdots\ |\ x \ |\ y \ |\ z \ |\ \cdots
      & \hspace{1em} &
      \clAtom & ::= & \clVar \ |\ \mathit{Const}
      \end{array}
      }
      \\
      \multicolumn{7}{c}{
      \begin{array}{rclclclcl}
        \clExp & ::= & \clAtom
        & | & \clExp\ +\ \clExp
        & | & \clExp\ -\ \clExp
        & | & \clExp\ \times\ \clExp
      \end{array}
      }
      \\
      \clAPred & ::= &
        \clExp = \clExp
      & | & \clExp \equiv \clExp \bmod \clExp
      & | & \clAPred \wedge \clAPred
      \\
      \clRPred & ::= &
      \multicolumn{3}{l}{
        \begin{array}{lclclcl}
          \clExp = \clExp
          & \hspace{2em}|\hspace{1em} & \clExp < \clExp
          & \hspace{2em}|\hspace{1em} & \neg \clRPred
        \end{array}
      }
      & | & \clRPred \wedge \clRPred
      \\
      \mathit{Inst} & ::= &
            \clMov\ \clVar\ \clAtom
      & | & \clCmov\ \clVar\ \clVar\
            \clAtom\ \clAtom
%      & | & \clCast\ \clVar\mytextsf{@}\mathit{Type}\ \clAtom
      & | & \clVPCast\ \clVar\mytextsf{@}\mathit{Type}\ \clAtom\\
      & | & \clUShl\ \clVar\ \clAtom\ \clNum
      & | & \clUAdd\ \clVar\ \clAtom\ \clAtom
      & | & \clUAdds\ \clVar\ \clVar\
            \clAtom\ \clAtom\\
      & | & \clSShl\ \clVar\ \clAtom\ \clNum
      & | & \clUAdc\ \clVar\ \clAtom\
            \clAtom\ \clAtom
      & | & \clUAdcs\ \clVar\ \clVar\
            \clAtom\ \clAtom\ \clAtom\\
      & | & \clUJoin\ \clVar\ \clAtom\ \clAtom
      & | & \clSAdd\ \clVar\ \clAtom\ \clAtom
      & | & \clSAdds\ \clVar\ \clVar\
            \clAtom\ \clAtom\\
      & | & \clSJoin\ \clVar\ \clAtom\ \clAtom
      & | & \clSAdc\ \clVar\ \clAtom\
            \clAtom\ \clAtom
      & | & \clSAdcs\ \clVar\ \clVar\
            \clAtom\ \clAtom\ \clAtom\\
      & | & \clUSplit\ \clVar\ \clVar\
            \clAtom\ \clNum
      & | & \clUSub\ \clVar\ \clAtom\ \clAtom
      & | & \clUSubs\ \clVar\ \clVar\
            \clAtom\ \clAtom\\
      & | & \clSSplit\ \clVar\ \clVar\
            \clAtom\ \clNum
      & | & \clUSbb\ \clVar\ \clAtom\
            \clAtom\ \clAtom
      & | & \clUSbbs\ \clVar\ \clVar\
            \clAtom\ \clAtom\ \clAtom\\
      & | & \clAssert\ \clAPred \clConj \clRPred
      & | & \clSSub\ \clVar\ \clAtom\ \clAtom
      & | & \clSSubs\ \clVar\ \clVar\
            \clAtom\ \clAtom\\
      & | & \clAssume\ \clAPred \clConj \clRPred
      & | & \clSSbb\ \clVar\ \clAtom\
            \clAtom\ \clAtom
      & | & \clSSbbs\ \clVar\ \clVar\
            \clAtom\ \clAtom\ \clAtom\\
      & | & \clUMul\ \clVar\ \clAtom\ \clAtom
      & | & \clUMull\ \clVar\ \clVar\
            \clAtom\ \clAtom
      & | & \clUCShl\ \clVar\ \clVar\
            \clAtom\ \clAtom\ \clNum\\
      & | & \clSMul\ \clVar\ \clAtom\ \clAtom
      & | & \clSMull\ \clVar\ \clVar\
            \clAtom\ \clAtom
      & | & \clSCShl\ \clVar\ \clVar\
            \clAtom\ \clAtom\ \clNum\\
      \mathit{Decl} & ::= &
        \multicolumn{2}{l}{\mathit{Type}\ \clVar}
      &&
      \multicolumn{2}{l}{
         \mathit{Prog} \hspace{2ex} ::=\hspace{2ex}
         \mathit{Decl}^*\ \mathit{Inst}^*
      }
    \end{array}
  \end{math}
  \caption{\cryptoline Syntax}
  \label{figure:cryptoline-syntax}
\end{figure*}

\cryptoline is a domain spceific language for the specification and
verification of cryptographic assembly programs. In addition to
arithmetic instructions, new instructions are added to model bitwise
operations and facilitate efficient verification.
Figure~\ref{figure:cryptoline-syntax} gives the syntax of \cryptoline.

A constant in \cryptoline must specify its type to have bit-accurate
semantics. For instance, $\mytextsf{0@uint32}$ and $\mytextsf{0@uint64}$
represent different bit strings. It is ambiguious to write
$\mytextsf{0}$ for the constant $0$ in \cryptoline. Variables in
\cryptoline are also typed. An \emph{atom} is either a variable or a
constant.

The $\clMov\ r\ x$ instruction assigns $r$ the value of atom $x$ where
$r$ and $x$ are of the same type. The conditional move instruction
$\clCmov\ r\ c\ x\ y$ assigns $x$ or $y$ to $r$ if the bit variable
$c$ is $1$ or $0$ respectively. Type coercion is explicit in
\cryptoline. The $\clVPCast\ r\mytextsf{@sint64}\ \mytextsf{0@uint64}$
converts the unsigned 64-bit constant $0$ to the signed 64-bit
constant $0$ and assigns it to the variable $r$ of type $\mytextsf{sint64}$.

Most \cryptoline instructions have signed and unsigned variants. For
instance, the $\clUAdd\ r\ x\ y$ instruction assigns the sum of
of \emph{unsigned} atoms $x$ and $y$ to the \emph{unsigned} variable
$r$. The $\clSAdd\ r\ x\ y$ on the other hand assigns the sum of the
values of \emph{signed} atoms $x$ and $y$ to the \emph{signed}
variable $r$. Carry flags are also explicit in \cryptoline. In
addition to assigning $r$ the sum of unsigned atoms $x$ and $y$, the
$\clUAdds\ c\ r\ x\ y$ instruction sets the carry bit $c$ to $1$ when
carrying occurs. The unsigned addition with carry instruction
$\clUAdc\ r\ x\ y\ d$ assigns the sum of unsigned atoms $x, y$ and the
carry bit $d$ to $r$. Similarly, $\clUAdcs\ c\ r\ x\ y\ d$ is the
unsigned carrying addition with carry instruction in \cryptoline.
Corresponding subtraction instructions are allowed as well. For
instance, the signed subtraction with borrow instruction $\clSSub\ r\
x\ y\ d$ assigns the signed variable $r$ to $x - y - d$ with signed
atoms $x$ and $y$. Signed and unsigned variants of full- and
half-multiplication instructions are supported in \cryptoline. For
instance, $\clSMul\ r\ x\ y$ assigns $r$ the lower half bits of the
product of signed atoms $x$ and $y$.

A handful of bitwise operations are added to \cryptoline to facilitate
efficient verification. The $\clUShl\ r\ x\ n$ instruction shifts the
unsigned atom $x$ to the left by $n$ bits and puts the result in
$r$. The $\clUJoin\ r\ x\ y$ instruction assigns $r$ the concatenation
of the bits of unsigned atoms $x$ and $y$. The $\clUSplit\
\mathit{hi}\ \mathit{lo}\ x\ n$ instruction splits the bits of the
unsigned atom $x$ into two: the low $n$ bits are put in the variable
$\mathit{lo}$; and remaining higher bits are put in $\mathit{hi}$.
The $\clUCShl\ \mathit{hi}\ \mathit{lo}\ x\ y\ n$ instruction first
shifts the concatenation of bits of the unsigned atoms $x$ and $y$ to
the left by $n$ bits. The shifted concatenation is then splitted into
two halves. The higher half is put to $\mathit{hi}$. The lower half is
shifted to the right by $n$ bits and then put to $\mathit{lo}$. Some
of these instructions do not directly correspond to assembly
instructions. Rather, they are widely used programming patterns found
in cryptographic codes. \cryptoline supports these special
instructions to simplify code specifications across different
architectures.

A couple of \cryptoline instructions are added for verification
purposes. The $\clAssert\ p \clConj q$ instruction checks if the
property $p \wedge q$ holds for all possible executions. It is an
error if the property fails at any execution. The $\clAssume\ p
\clConj q$ instruction on the other hand assumes $p \wedge q$ for all
possible executions. The syntax $p \clConj q$ delimits the algebraic
property $p$ and the range property $q$. An algebraic property is a
conjunction of equations and modulo equations over expressions. A
range property on the other hand is an arbitrary Boolean combination
of equations and comparisons.

% example

\begin{figure}
  \centering
  \begin{tabular}{l}
    $\clUSubs$ b t in N;\\
    $\clUSbbs$ dontcare mask 0$\mytextsf{@uint64}$ $\mytextsf{0@uint64}$ b;\\
    $\clAnd$ r0$\mytextsf{@uint64}$ mask in;\\
    $\clNot$ nmask$\mytextsf{@uint64}$ mask;\\
    $\clAnd$ r1$\mytextsf{@uint64}$ nmask t;\\
    $\clOr$ r$\mytextsf{@uint64}$ r0 r1;\\
    $\clAssert$ $\clTrue$ $\clConj$ r = in $\wedge$ r = in - N;
  \end{tabular}
  \caption{Constant-time Selection}
  \label{figure:cryptoline-example}
\end{figure}

Figure~\ref{figure:cryptoline-example} gives a.

% verification condition generation

