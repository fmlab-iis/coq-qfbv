
\begin{figure*}
  \centering
  \begin{math}
    \begin{array}{rclclcl}
      \multicolumn{7}{c}{
      \begin{array}{rclcrcl}
        \mathit{Num} & ::= & \cdots \ |\ \mathsf{-2} \ |\ \mathsf{-1}
                             \ |\ \mathsf{0} \ |\ \mathsf{1} \ |\
                             \mathsf{2} \ |\  \cdots
        & \hspace{1em} &
        \mathit{Type} & ::= & \mathsf{uint32}\ |\ \mathsf{sint32}\ |\
                              \mathsf{uint64}\ |\ \mathsf{sint64}\ |\ \cdots
      \end{array}
      }\\
      \multicolumn{7}{c}{
      \begin{array}{rclcrclcrcl}
      \mathit{Const} & ::= & \mathit{Num}\mathsf{@}\mathit{Type}
      & \hspace{1em} &
      \mathit{Var} & ::= & \cdots\ |\ x \ |\ y \ |\ z \ |\ \cdots
      & \hspace{1em} &
      \mathit{Atom} & ::= & \mathit{Var} \ |\ \mathit{Const}
      \end{array}
      }
      \\
      \multicolumn{7}{c}{
      \begin{array}{rclclclcl}
        \mathit{Exp} & ::= & \mathit{Atom}
        & | & \mathit{Exp}\ +\ \mathit{Exp}
        & | & \mathit{Exp}\ -\ \mathit{Exp}
        & | & \mathit{Exp}\ \times\ \mathit{Exp}
      \end{array}
      }
      \\
      \mathit{APred} & ::= &
        \mathit{Exp} = \mathit{Exp}
      & | & \mathit{Exp} \equiv \mathit{Exp} \bmod \mathit{Exp}
      & | & \mathit{APred} \wedge \mathit{APred}
      \\
      \mathit{RPred} & ::= &
        \mathit{Exp} = \mathit{Exp}
      & | & \mathit{Exp} < \mathit{Exp}
      & | & \mathit{RPred} \wedge \mathit{RPred}
      \\
      \mathit{Inst} & ::= &
            \clMov\ \mathit{Var}\ \mathit{Atom}
      & | & \clCmov\ \mathit{Var}\ \mathit{Var}\
            \mathit{Atom}\ \mathit{Atom}
%      & | & \clCast\ \mathit{Var}\mathsf{@}\mathit{Type}\ \mathit{Atom}
      & | & \clVPCast\ \mathit{Var}\mathsf{@}\mathit{Type}\ \mathit{Atom}\\
      & | & \clUShl\ \mathit{Var}\ \mathit{Atom}\ \mathit{Num}
      & | & \clUAdd\ \mathit{Var}\ \mathit{Atom}\ \mathit{Atom}
      & | & \clUAdds\ \mathit{Var}\ \mathit{Var}\
            \mathit{Atom}\ \mathit{Atom}\\
      & | & \clSShl\ \mathit{Var}\ \mathit{Atom}\ \mathit{Num}
      & | & \clUAdc\ \mathit{Var}\ \mathit{Atom}\
            \mathit{Atom}\ \mathit{Atom}
      & | & \clUAdcs\ \mathit{Var}\ \mathit{Var}\
            \mathit{Atom}\ \mathit{Atom}\ \mathit{Atom}\\
      & | & \clUJoin\ \mathit{Var}\ \mathit{Atom}\ \mathit{Atom}
      & | & \clSAdd\ \mathit{Var}\ \mathit{Atom}\ \mathit{Atom}
      & | & \clSAdds\ \mathit{Var}\ \mathit{Var}\
            \mathit{Atom}\ \mathit{Atom}\\
      & | & \clSJoin\ \mathit{Var}\ \mathit{Atom}\ \mathit{Atom}
      & | & \clSAdc\ \mathit{Var}\ \mathit{Atom}\
            \mathit{Atom}\ \mathit{Atom}
      & | & \clSAdcs\ \mathit{Var}\ \mathit{Var}\
            \mathit{Atom}\ \mathit{Atom}\ \mathit{Atom}\\
      & | & \clUSplit\ \mathit{Var}\ \mathit{Var}\
            \mathit{Atom}\ \mathit{Num}
      & | & \clUSub\ \mathit{Var}\ \mathit{Atom}\ \mathit{Atom}
      & | & \clUSubs\ \mathit{Var}\ \mathit{Var}\
            \mathit{Atom}\ \mathit{Atom}\\
      & | & \clSSplit\ \mathit{Var}\ \mathit{Var}\
            \mathit{Atom}\ \mathit{Num}
      & | & \clUSbb\ \mathit{Var}\ \mathit{Atom}\
            \mathit{Atom}\ \mathit{Atom}
      & | & \clUSbbs\ \mathit{Var}\ \mathit{Var}\
            \mathit{Atom}\ \mathit{Atom}\ \mathit{Atom}\\
      & | & \clAssert\ \mathit{APred} \clConj \mathit{RPred}
      & | & \clSSub\ \mathit{Var}\ \mathit{Atom}\ \mathit{Atom}
      & | & \clSSubs\ \mathit{Var}\ \mathit{Var}\
            \mathit{Atom}\ \mathit{Atom}\\
      & | & \clAssume\ \mathit{APred} \clConj \mathit{RPred}
      & | & \clSSbb\ \mathit{Var}\ \mathit{Atom}\
            \mathit{Atom}\ \mathit{Atom}
      & | & \clSSbbs\ \mathit{Var}\ \mathit{Var}\
            \mathit{Atom}\ \mathit{Atom}\ \mathit{Atom}\\
      & | & \clUMul\ \mathit{Var}\ \mathit{Atom}\ \mathit{Atom}
      & | & \clUMull\ \mathit{Var}\ \mathit{Var}\
            \mathit{Atom}\ \mathit{Atom}
      & | & \clUCShl\ \mathit{Var}\ \mathit{Var}\
            \mathit{Atom}\ \mathit{Atom}\ \mathit{Num}\\
      & | & \clSMul\ \mathit{Var}\ \mathit{Atom}\ \mathit{Atom}
      & | & \clSMull\ \mathit{Var}\ \mathit{Var}\
            \mathit{Atom}\ \mathit{Atom}
      & | & \clSCShl\ \mathit{Var}\ \mathit{Var}\
            \mathit{Atom}\ \mathit{Atom}\ \mathit{Num}\\
      \mathit{Decl} & ::= &
        \multicolumn{2}{l}{\mathit{Type}\ \mathit{Var}}
      &&
      \multicolumn{2}{l}{
         \mathit{Prog} \hspace{2ex} ::=\hspace{2ex}
         \mathit{Decl}^*\ \mathit{Inst}^*
      }
    \end{array}
  \end{math}
  \caption{\cryptoline Syntax}
  \label{figure:cryptoline-syntax}
\end{figure*}

\cryptoline is a domain spceific language for the specification and
verification of cryptographic assembly programs. In addition to
arithmetic instructions, new instructions are added to model bitwise
operations and facilitate efficient verification.
Figure~\ref{figure:cryptoline-syntax} gives the syntax of \cryptoline.

A constant in \cryptoline must specify its type to have bit-accurate
semantics. For instance, $\mathsf{0@uint32}$ and $\mathsf{0@uint64}$
represent different bit strings. It is ambiguious to write
$\mathsf{0}$ for the constant $0$ in \cryptoline. An \emph{atom} is
either a variable or a constant.
